\newpage 
\section*{Parameter values - original}
[Copied from another file as an example of what we are looking for.]

A few collected parameter values are given in Table~\ref{tab:params}. In \cite{owen1997pattern}, the random motility and chemotaxis parameters for macrophages are obtained from previously published (Boyden chamber)  experimental data, giving values in the range of 5 $\times 10^{-15}$m$^2$/s. (See their Table 1.) Those values were used for modeling macrophage-tumor interactions in \cite{owen1997pattern}. Those values are cited for a model of glioma in \cite{khajanchi2021spatiotemporal}. They are converted to more convenient units below.

In \cite{khajanchi2021spatiotemporal}, we also find a typical proliferation rate of macrophages $\approx 0.3307$/h, and a carrying capacity (typical density) of  $10^6$ cells (domain size unspecified). Typical macrophage density: According to \cite{gupta2006spatiotemporal} there are 0.1 (normal) to about 1.0-1.2 (in injured tissue)  macrophages $\times 10^{-4}$ per $\mu$m$^2$. This buildup typically takes about 7 days. This leads to an estimate for recruitment rate of about $10^{-4}/(6\times 10^5) \approx 10^{-10}$ cells /$\mu$m$^2$/s. But in our 1D model, we need the square-root of the above domain, so obtaining a basal recruitment rate of around $a_0\approx 10^{-5}$ cells/$\mu$m/s.
Macrophages can survive in tissue ``macrophages survive in tissue for weeks or months'' according to \cite{owen1997pattern}, whereas other cells have a turnover time of days.


The paper by \cite{masur1996myofibroblasts} provides some information about myofibroblast ``recruitment'' (transdifferentiation from fibroblasts), stating that (rabit corneal) fibroblasts at a density of 5 or 500 cells/mm$^2$ produced 80\% or 10\% myofibroblasts after 5-7 days (in vitro experiment). According to \cite{gascard2016carcinoma} cancer associated fibroblasts (CAFs) can occupy about 80\% of a tumor. Taking a typical cell diameter ($\approx 10\mu$m), and 50\% of cells as fibroblasts in the tissue, we have a typical tissue fibroblast density of 0.05 cells/$\mu$m (in 1D). Suppose half of these transdifferntiate into myofibroblasts in 7 days. Then we have a ``basal myofibroblast recruitment rate'' of  $0.025/(7$ days) which is $\approx 4 \times 10^{-8}$ cells/s. [Note: myofibroblasts can be much larger than fibroblasts, about $50\mu$m long and $25\mu$m wide, \cite{masur1996myofibroblasts}.] They persist in cultures for about 3-7 days, so we can estimate their decay rate as $\delta_m\approx 1/7$(/day), which is roughly $10^{-6}$/s.

What we also need to estimate is the macrophage-induced rate of myofibroblast recruitment $b$ (likely dependent on cytokines secreted by macrophages etc.). However, we could ball-park estimate that $b \approx 2-10 \times b_0$. 

We refer to \cite{pakshir2019dynamic} for parameters of stress-related macrophage recruitment rate. According to this paper, macrophages are attracted to a stress field of myofibroblasts within a radius of around 600 $\mu$m at migration speeds of 0.5-1.4 $\mu$m/min. Taking a mean speed of 1$\mu$m/min resulting in a recruitment rate constant of roughly $1/(600\cdot 60)\approx 3 \times 10^{-5} $/s. This is still not what we need for the parameter $a_1$ (or $a_2$)  in our model's function $a(\sigma)$ as we need information about number (or density) of cells recruited per unit stress per unit time. 

We may be able to get some information about the stress field induced by myofibroblasts (i.e. obtain $\alpha f(m)$ for our model) from the paper \cite{pakshir2019dynamic}.

\begin{table}[ht]
    \centering
    \begin{tabular}{|l|l|c|c|l|} \hline
     Parameter & Meaning &value &units & Source  \\ \hline
     
      $D_\phi$ & Macrophage random motility & 0.005 & $\mu$m$^2$/s &in cancer \cite{owen1997pattern,khajanchi2021spatiotemporal} \\
       $D_m$ & Myofibroblast random motility & & $\mu$m$^2$/s & \\
      
      $a_0$& Basal macrophage recruitment rate&$10^{-5}$& cells $\mu$m$^{-1}$/s& estimated from \cite{gupta2006spatiotemporal}\\
      
    $b_0$&Basal myofibroblast recruitment rate & $4 \times 10^{-8}$& cells/s& estimated from \cite{masur1996myofibroblasts,gascard2016carcinoma} \\
    $b$&macrophage-induced myof recruit rate& ~$\approx 2-10\times b_0$& cells/s&very rough guess\\

    
    $\delta_m$&Myofibroblast turnover rate& $10^{-6}$& 1/s& estimated from \cite{masur1996myofibroblasts}\\

    $\delta_\phi$&Macrophage turnover rate& $10^{-7}$&1/s& order of magn slower \cite{owen1997pattern}\\

    &Normal macrophage density & &cells/$\mu$m$^3$ &\\
 
      &Inflammed macrophage density & &cells/$\mu$m$^3$ &(or ratio of these)\\
      
      &Typical collagen density & 
        &Amt/$\mu$m$^3$& \\

    &Typical collagen turnover rate&
        & 1/time & \\
        
      &Typical mofib density& 
      &cells/$\mu$m$^3$ &\\
      
     &Typical myofib traction force& &pN/cell& \\   \hline
    \end{tabular}
    \caption{Typical ball-park estimates for some parameters in the model. \LEK{Many of these will not be easy to find, and will require ball-park estimates at best.}}
    \label{tab:params}
\end{table}
